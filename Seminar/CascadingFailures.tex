%----------------------------------------------------------------------------------------
%	PACKAGES AND OTHER DOCUMENT CONFIGURATIONS
%----------------------------------------------------------------------------------------

% \documentclass[paper=a4, fontsize=12pt]{article}
\documentclass[a4paper,12pt]{article}
\usepackage{fancyhdr} % Required for custom headers
\usepackage{lastpage} % Required to determine the last page for the footer
\usepackage{extramarks} % Required for headers and footers
\usepackage{graphicx} % Required to insert images
\usepackage[T1]{fontenc}
%\usepackage{fbb}
%\usepackage[cmintegrals,cmbraces]{newtxmath}

\usepackage[T1]{fontenc}
\usepackage[utf8]{inputenc}
\usepackage{amsmath} % Required for math mode
\usepackage{textcomp} % Required for text symbols
\usepackage{amssymb} % Required for math symbols
\usepackage[section]{placeins}
\usepackage{caption}
\usepackage{subfig}
\usepackage{bm}
\usepackage{multirow}
\usepackage[linkbordercolor={1 0.5 0.4}]{hyperref}


% \usepackage[title,titletoc,toc]{appendix}\
% \usepackage[title,toc]{appendix}
% Margins
\topmargin=-0.45in
\evensidemargin=-0.1in
\oddsidemargin=-0.1in
\textwidth=6.5in
\textheight=9.4in
\headsep=0.25in 

\linespread{1.1} % Line spacing

% Set up the header and footer
\pagestyle{fancy}
\lhead{\hmwkAuthorName} % Top left header
% \chead{\ \hmwkClassInstructor\ \hmwkClassTime \hmwkTitle} % Top center header
\rhead{WS2021}
% \rhead{\firstxmark} % Top right header
\lfoot{\lastxmark} % Bottom left footer
\cfoot{} % Bottom center footer
\rfoot{Page\ \thepage\ of\ \pageref{LastPage}} % Bottom right footer
\renewcommand\headrulewidth{0.4pt} % Size of the header rule
\renewcommand\footrulewidth{0.4pt} % Size of the footer rule

\setlength\parindent{0pt} % Removes all indentation from paragraphs

%---------------------------------------------------------------------------------------
%	NAME AND CLASS SECTION
%----------------------------------------------------------------------------------------

\newcommand{\hmwkTitle}{Cascading Failures in networks (power grid)} % Assignment title
\newcommand{\hmwkDueDate}{} % Due date
\newcommand{\hmwkClass}{} % Course/class
\newcommand{\hmwkClassTime}{} % Class/lecture time 10:30am
\newcommand{\hmwkClassInstructor}{} % Teacher/lecturer Jones
\newcommand{\hmwkAuthorName}{Seminar on Network Science} % Your name

%----------------------------------------------------------------------------------------
%	TITLE PAGE
%----------------------------------------------------------------------------------------
\newcommand{\horrule}[1]{\rule{\linewidth}{#1}} % Create horizontal rule command with 1 argument of height

\title{
\vspace{2in}
\huge{\textsc{HU Berlin}\\\vspace{0.3cm}\hmwkClass}\\
\horrule{0.5pt} \\[0.4cm] % Thin top horizontal rule
\huge{\textmd{\textbf{\ \hmwkTitle}}}\\
\horrule{2pt} \\[0.5cm] % Thick bottom horizontal rule
% \normalsize\vspace{0.1in}\LARGE{\textit{\hmwkDueDate}}\\%Due\ on\ 
% \vspace{0.1in}\LARGE{\textit{\hmwkClassInstructor\ \hmwkClassTime}}
\vspace{3in}
}

\author{\Large{\textbf{\hmwkAuthorName}}}
\date{\today} % Insert date here if you want it to appear below your name

%----------------------------------------------------------------------------------------

\begin{document}

% \maketitle

%----------------------------------------------------------------------------------------
%	TABLE OF CONTENTS
%----------------------------------------------------------------------------------------

%\setcounter{tocdepth}{1} % Uncomment this line if you don't want subsections listed in the ToC

% \newpage
% \tableofcontents
% \newpage
\vspace*{1mm}
\begin{center}
    {\LARGE\textbf{\hmwkTitle}}\\
   \hmwkDueDate
\end{center}

\subsection*{Literature}
\textbf{Small vulnerable sets determine large network cascades in power grids}\\
Yang Yang, Takashi Nishikawa, Adilson E. Motter\\
Science 358, 886 (2017)\\
\\
\textbf{Cascading Failures as Continuous Phase-Space Transitions}\\
Yang Yang, Adilson E. Motter\\
PRL 119, 248302 (2017)\\
\\
\textbf{Emergent Failures and Cascades in Power Grids: A Statistical Physics Perspective}\\
Tommaso Nesti, Alessandro Zocca, Bert Zwart\\
Phys. Rev. Lett. 120, 258301 (2018) \\

Any other papers addressing the similar topic can be used/presented.

\subsubsection*{The following concepts should be explained:}
\small{
\begin{itemize}
\item Select one of the suggested papers or find similar papers addressing the disturbance propagation in power grid. You could also consider cascading failures in other social networks.
\item Do the similar simulations for synthetic networks. If you are interested in this topic we could have a talk for the simulation part.
\end{itemize}}

\end{document}
