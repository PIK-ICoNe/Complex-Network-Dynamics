%----------------------------------------------------------------------------------------
%	PACKAGES AND OTHER DOCUMENT CONFIGURATIONS
%----------------------------------------------------------------------------------------

% \documentclass[paper=a4, fontsize=12pt]{article}
\documentclass[a4paper,12pt]{article}
\usepackage{fancyhdr} % Required for custom headers
\usepackage{lastpage} % Required to determine the last page for the footer
\usepackage{extramarks} % Required for headers and footers
\usepackage{graphicx} % Required to insert images
\usepackage[T1]{fontenc}
%\usepackage{fbb}
%\usepackage[cmintegrals,cmbraces]{newtxmath}

\usepackage[T1]{fontenc}
\usepackage[utf8]{inputenc}
\usepackage{amsmath} % Required for math mode
\usepackage{textcomp} % Required for text symbols
\usepackage{amssymb} % Required for math symbols
\usepackage[section]{placeins}
\usepackage{caption}
\usepackage{subfig}
\usepackage{bm}
\usepackage{multirow}
\usepackage[linkbordercolor={1 0.5 0.4}]{hyperref}


% \usepackage[title,titletoc,toc]{appendix}\
% \usepackage[title,toc]{appendix}
% Margins
\topmargin=-0.45in
\evensidemargin=-0.1in
\oddsidemargin=-0.1in
\textwidth=6.5in
\textheight=9.4in
\headsep=0.25in 

\linespread{1.1} % Line spacing

% Set up the header and footer
\pagestyle{fancy}
\lhead{\hmwkAuthorName} % Top left header
% \chead{\ \hmwkClassInstructor\ \hmwkClassTime \hmwkTitle} % Top center header
\rhead{WS2021}
% \rhead{\firstxmark} % Top right header
\lfoot{\lastxmark} % Bottom left footer
\cfoot{} % Bottom center footer
\rfoot{Page\ \thepage\ of\ \pageref{LastPage}} % Bottom right footer
\renewcommand\headrulewidth{0.4pt} % Size of the header rule
\renewcommand\footrulewidth{0.4pt} % Size of the footer rule

\setlength\parindent{0pt} % Removes all indentation from paragraphs

%---------------------------------------------------------------------------------------
%	NAME AND CLASS SECTION
%----------------------------------------------------------------------------------------

\newcommand{\hmwkTitle}{Multistability of phase-locking in Kuramoto model} % Assignment title
\newcommand{\hmwkDueDate}{} % Due date
\newcommand{\hmwkClass}{} % Course/class
\newcommand{\hmwkClassTime}{} % Class/lecture time 10:30am
\newcommand{\hmwkClassInstructor}{} % Teacher/lecturer Jones
\newcommand{\hmwkAuthorName}{Seminar on Network Science} % Your name

%----------------------------------------------------------------------------------------
%	TITLE PAGE
%----------------------------------------------------------------------------------------
\newcommand{\horrule}[1]{\rule{\linewidth}{#1}} % Create horizontal rule command with 1 argument of height

\title{
\vspace{2in}
\huge{\textsc{HU Berlin}\\\vspace{0.3cm}\hmwkClass}\\
\horrule{0.5pt} \\[0.4cm] % Thin top horizontal rule
\huge{\textmd{\textbf{\ \hmwkTitle}}}\\
\horrule{2pt} \\[0.5cm] % Thick bottom horizontal rule
% \normalsize\vspace{0.1in}\LARGE{\textit{\hmwkDueDate}}\\%Due\ on\ 
% \vspace{0.1in}\LARGE{\textit{\hmwkClassInstructor\ \hmwkClassTime}}
\vspace{3in}
}

\author{\Large{\textbf{\hmwkAuthorName}}}
\date{\today} % Insert date here if you want it to appear below your name

%----------------------------------------------------------------------------------------

\begin{document}

% \maketitle

%----------------------------------------------------------------------------------------
%	TABLE OF CONTENTS
%----------------------------------------------------------------------------------------

%\setcounter{tocdepth}{1} % Uncomment this line if you don't want subsections listed in the ToC

% \newpage
% \tableofcontents
% \newpage
\vspace*{1mm}
\begin{center}
    {\LARGE\textbf{\hmwkTitle}}\\
   \hmwkDueDate
\end{center}

\subsection*{Literature}
\textbf{Multistability of phase-locking in equal-frequency Kuramoto models on planar graphs}\\
Robin Delabays, Tommaso Coletta, Philippe Jacquod \\
J. Math. Phys. 58, 032703 (2017)\\
\\
\textbf{Algebraic geometrization of the Kuramoto model: Equilibria and stability analysis}\\
Dhagash Mehta, Noah S. Daleo, Florian Dörfler, et al\\
Chaos 25, 053103 (2015)\\
\\
\textbf{Cycle flows and multistability in oscillatory networks}\\
Debsankha Manik, Marc Timme, and Dirk Witthaut\\
Chaos 27, 083123 (2017)\\

Any other papers addressing the similar topic can be used/presented.

\subsubsection*{The following concepts should be explained:}
\small{
\begin{itemize}
\item Select one of the suggested papers or find similar papers addressing the multistability of phase-locking in first or second order of KM.
\item Study and present the analytical approach to demonstrate the multistability approach.
\item Do the simulation to show the existing of multistability.
\end{itemize}}

\end{document}
