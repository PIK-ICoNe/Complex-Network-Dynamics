%----------------------------------------------------------------------------------------
%	PACKAGES AND OTHER DOCUMENT CONFIGURATIONS
%----------------------------------------------------------------------------------------

% \documentclass[paper=a4, fontsize=12pt]{article}
\documentclass[a4paper,12pt]{article}
\usepackage{fancyhdr} % Required for custom headers
\usepackage{lastpage} % Required to determine the last page for the footer
\usepackage{extramarks} % Required for headers and footers
\usepackage{graphicx} % Required to insert images
\usepackage[T1]{fontenc}
%\usepackage{fbb}
%\usepackage[cmintegrals,cmbraces]{newtxmath}

\usepackage[T1]{fontenc}
\usepackage[utf8]{inputenc}
\usepackage{amsmath} % Required for math mode
\usepackage{textcomp} % Required for text symbols
\usepackage{amssymb} % Required for math symbols
\usepackage[section]{placeins}
\usepackage{caption}
\usepackage{subfig}
\usepackage{bm}
\usepackage{multirow}
\usepackage[linkbordercolor={1 0.5 0.4}]{hyperref}


% \usepackage[title,titletoc,toc]{appendix}\
% \usepackage[title,toc]{appendix}
% Margins
\topmargin=-0.45in
\evensidemargin=-0.1in
\oddsidemargin=-0.1in
\textwidth=6.5in
\textheight=9.4in
\headsep=0.25in 

\linespread{1.1} % Line spacing

% Set up the header and footer
\pagestyle{fancy}
\lhead{\hmwkAuthorName} % Top left header
% \chead{\ \hmwkClassInstructor\ \hmwkClassTime \hmwkTitle} % Top center header
\rhead{WS2021}
% \rhead{\firstxmark} % Top right header
\lfoot{\lastxmark} % Bottom left footer
\cfoot{} % Bottom center footer
\rfoot{Page\ \thepage\ of\ \pageref{LastPage}} % Bottom right footer
\renewcommand\headrulewidth{0.4pt} % Size of the header rule
\renewcommand\footrulewidth{0.4pt} % Size of the footer rule

\setlength\parindent{0pt} % Removes all indentation from paragraphs

%---------------------------------------------------------------------------------------
%	NAME AND CLASS SECTION
%----------------------------------------------------------------------------------------

\newcommand{\hmwkTitle}{Probabilistic Stability in Power Grids} % Assignment title
\newcommand{\hmwkDueDate}{} % Due date
\newcommand{\hmwkClass}{} % Course/class
\newcommand{\hmwkClassTime}{} % Class/lecture time 10:30am
\newcommand{\hmwkClassInstructor}{} % Teacher/lecturer Jones
\newcommand{\hmwkAuthorName}{Seminar on Network Science} % Your name

%----------------------------------------------------------------------------------------
%	TITLE PAGE
%----------------------------------------------------------------------------------------
\newcommand{\horrule}[1]{\rule{\linewidth}{#1}} % Create horizontal rule command with 1 argument of height

\title{
\vspace{2in}
\huge{\textsc{HU Berlin}\\\vspace{0.3cm}\hmwkClass}\\
\horrule{0.5pt} \\[0.4cm] % Thin top horizontal rule
\huge{\textmd{\textbf{\ \hmwkTitle}}}\\
\horrule{2pt} \\[0.5cm] % Thick bottom horizontal rule
% \normalsize\vspace{0.1in}\LARGE{\textit{\hmwkDueDate}}\\%Due\ on\ 
% \vspace{0.1in}\LARGE{\textit{\hmwkClassInstructor\ \hmwkClassTime}}
\vspace{3in}
}

\author{\Large{\textbf{\hmwkAuthorName}}}
\date{\today} % Insert date here if you want it to appear below your name

%----------------------------------------------------------------------------------------

\begin{document}

% \maketitle

%----------------------------------------------------------------------------------------
%	TABLE OF CONTENTS
%----------------------------------------------------------------------------------------

%\setcounter{tocdepth}{1} % Uncomment this line if you don't want subsections listed in the ToC

% \newpage
% \tableofcontents
% \newpage
\vspace*{1mm}
\begin{center}
    {\LARGE\textbf{\hmwkTitle}}\\
   \hmwkDueDate
\end{center}

\subsection*{Literature}
\textbf{How dead ends undermine power grid stability.}\\
Menck, P. J., et al.\\
Nature communications 5.1 (2014): 1-8.\\
\url{https://www.nature.com/articles/ncomms4969}\\

\textbf{Survivability of deterministic dynamical systems.}\\
Hellmann, F., et al. \\
Scientific reports 6.1 (2016): 1-12.\\
\url{https://www.nature.com/articles/srep29654}\\

\textbf{Deciphering the imprint of topology on nonlinear dynamical network stability.}\\
Nitzbon, J., et al.\\
New Journal of Physics 19.3 (2017): 033029.\\
\url{https://iopscience.iop.org/article/10.1088/1367-2630/aa6321/pdf}
\\

Please also find additional literature.

\subsubsection*{Topic}

In order to investigate the non-linear behaviour of complicated network dynamical systems, it is often useful to study probabilistic properties of the system. A key example is the probability that a random perturbation at a node destabilizes the network.


\small{
\begin{itemize}
\item Explain the key notions of probabilistic stability
\item Implement an example
\end{itemize}}

\end{document}

